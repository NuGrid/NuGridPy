% *********** Document name and reference:
% Title of document
\renewcommand{\ndoctitle}{NUPYCEE} 
% Document category acronym 
\renewcommand{\ndocname}{NUPYCEE}                      
% svn dir
\renewcommand{\svndir}{svn://forum.astro.keele.ac.uk/utils/GCE/NUPYCEE}  
% Contributors to this document
\renewcommand{\ndoccontribs}{Christian Ritter, Benoit C\^ot\'e}

%%\chapter{\ndoctitle}   % uncomment this if compiled into book in docs directory
\ngtitle{\ndoctitle}     % uncomment this is compiled into chapter wrapper this directory


Document name: \ndocname \\
SVN directory: \svndir\\
Contributors: \ndoccontribs\\
Please send any feedback to critter@uvic.ca or bcote@uvic.ca. Thank you.


{  \textbf{Abstract:} %\slshape{}
%\begin{abstract}
\noindent The Nugrid-Python-Chemical-evolution environment, short NuPyCEE, provides
a framework to simulate the ejecta of simple stellar populations with the 1-zone code
Stellar Yields for Galactic Modeling Applications as well as galaxies
with the One-zone Model for the Evolution of Galaxies.
Various chemical evolution input parameter can be specified through a python-based interface.
Both tools provide extraction and plotting methods.
We also provide a web interface with limited capabilities which allows to download
yield tables containing the ejecta of SSP's. 
To compare the results to observational data the user can utilize the STELLAB module
which provides various spectroscopic abundance ratios of stars such as
Milky Way and dwarf galaxies.
%\end{abstract}
}

%##############################################################
%# Section: Introduction
%##############################################################


\section{Introduction}
\index{\ndocname}
Welcome to the \ndocname\ user Guide. The purpose of \ndocname\ is to 
provide the user with a framework to perform chemical evolution calculations.
The focus lies on simplification and user-friendliness.
The user can choose between the \textit{Stellar Yields for Galactic Modeling Applications} (SYGMA) module 
and the \textit{One-zone Model for the Evolution of Galaxies} (OMEGA) module.
SYGMA allows to follow the ejecta of simple stellar populations in a closed box while OMEGA deals with
the simulation of galaxies with inflows and outflows.
Both tools can be used in a similar manner through the ipython command line or
ipython notebooks.
%However included are low-mass, intermediate and massive star contributions, including
%SN2 and SN1a.

%This section will provide the user with a 
%tutorial and walk through of the tools contained within \ndocname\ and
%how a typical user would go about using and working with this python module.
%Note also that each SYGMA version provides a python-based function documentation (Sphinx) 
%in Sphinx\textunderscore DOC which can be accessed with the web brower through build/html/index.hmtl.

\subsection{Fastest possible instruction}

To download the code
you need to have git installed on your local machine.
Then download the code from github at

\url{https://github.com/NuGrid/NuPyCEE}

into your local directory via

\begin{verbatim}
git clone  https://github.com/NuGrid/NuPyCEE.git .
\end{verbatim}

All python dependencies can be installed via 

install\textunderscore dependencies.sh 

available in the root directory of NUPYCEE.

Start ipython in the NUPYCEE directory with

\begin{verbatim}
ipython -pylab -p numpy
\end{verbatim}

To run a SYGMA calculation:

\begin{verbatim}
import sygma as s
s1=s.sygma()
\end{verbatim}

That's it! To plot the total  mass ejected:

\begin{verbatim}
s1.plot_totmasses(source='all')	
\end{verbatim}

%More plot functions are available: s1.plot$<$tab$>$.
%Use a question mark after the plotting function
%for more infos.

To run OMEGA:

\begin{verbatim}
import omega as o
o1=o.omega()
\end{verbatim}

\begin{verbatim}
o1.plot_spectro(xaxis='[Fe/H]',yaxis='[O/Fe]')
\end{verbatim}

Now you might want to compare the prediction with
observations through the STELLAB module.

\begin{verbatim}
import stellab as st
st1=st.stellab()
st1.plot_spectro(xaxis='[Fe/H]', yaxis='[O/Fe]')
\end{verbatim}




%Check out the documentation in Section \ref{sect_omega}. 

\subsection{Preparing the environment}%Tutorial and Walkthough}


All python dependencies can be installed via 

install\textunderscore dependencies.sh 

available in the root directory of NUPYCEE

To take advantage of the interactive ipython environment 
SYGMA and OMEGA are both launched after starting
ipython with

\begin{verbatim}
ipython -pylab -p numpy
\end{verbatim}


%Make sure your PYTHONPATH variable is pointing to this directory.

\subsection{SYGMA}

\subsubsection{General concepts}

In order to follow the chemical enrichment via gas particles
in N-body simulations one has to describe the ejecta of material
from such star particles, essentially simple stellar populations. 
With SYGMA one can follow the ejecta of such populations via
diagrams and you can extract the composition of the ejecta via tables.


\subsubsection{Getting started}

If one wants to start SYGMA from outside the SYGMA directory
one has to specify the $SYGMADIR$ variable which
points to the SYGMA dir.

Import the module:

\begin{verbatim}
import sygma as s
\end{verbatim}

Next, we create an SSP instance, initiate a class instance $s1$ of SYGMA:


\begin{verbatim}
s1=sygma.sygma(iniZ=0.0001,tend=1e10,mgal=1e9)
\end{verbatim}

A single stellar population of mass $1\times10^{9}\msun$ and metal fraction
of $0.0001$ is created and evolved in time steps of $10^{8}$ years
to the final time of 1e10 years. Other not specified and hence
default parameter include for example the choice of the initial-mass function.
%range of $1\msun$ up to $30\msun$ is used. NuGrid tables 
%and SN1a yields from Thielemann86 are taken


One can check the definition of these input parameter online via
\url{http://nugrid.github.io/NuPyCEE/SPHINX/build/html/sygma.html}.


No yield tables are defined and hence the NuGrid tables as the
default choice were selected. These are the Set1 and Set1extension
yields which are
%\begin{verbatim}
%s1=sygma.sygma(alphaimf=2.35,sfr='input',iniZ=0.0001,dt=1e8,tend=1e10,mgal=1e9,
% table='yield_tables/isotope_yield_table.txt',
% sn1a_table='yield_tables/sn1a_t86.txt',
% bb_table='yield_tables/bb_walker91.txt')
%\end{verbatim} 
AGB and massive star yields for five different metallicities.
In this mode the user specifies $iniZ$
which can be either $Z=2e-2, 1e-2, 6e-3, 1e-3, 1e-4$ and $0.0$.
Scaled abundances for Z=1e-t65, 1e-6 will be available soon.
Currently for $Z=0$ the PopIII stars from \cite{heger:10} are set
as default.

\subsubsection{Providing your own yield tables}

Yield tables are available in the NUPYCEE subdirectory 
yield\textunderscore tables. Add your yield tables to
this directory and SYGMA will be able read the table
if you have specified the $table$ variable. Only
for table of Z=0 the variable $pop3\textunderscore table$ is used.
Both tables need yields specified in the SYGMA (and OMEGA)
yield input format. See for the structure the default table.
It is important to provide an initial abundance
file which has to match the number of species provided in the yield tables.
Provide the file in the iniAbu directory inside the directory yield\textunderscore tables.
The input variable with which the table file can be specified is $iniabu\textunderscore table$.
For the necessary structure see again the default choice of that variable.
%Another subdirectory,
%plotting\textunderscore scripts, contains various 
%plotting scripts. The latter use SYGMA's plotting functions.


\subsubsection{What do I get out?}


The output during execution shows some initial parameters,
e.g. the isotopes being part of chemical evolution,
as well as the time and metallicity evolution.
Also shown is the contribution time of stars of 
certain initial mass.



%For more examples, see the SYGMA ipython notebooks available
%at \\http://208.75.74.161:8888/tree/Publicnotebooks/intro\textunderscore SYGMA.

To analyze the run afterwards we provide a variety of plotting  functions
which are described in detail at \url{http://nugrid.github.io/NuPyCEE/SPHINX/build/html/sygma.html}.
In following we give some examples. The function plot\textunderscore mass() shows the evolution
of the yield in solar masses.


\begin{verbatim}
s1.plot_mass()
\end{verbatim}

With SYGMA you can look at different contributions
for AGB stars, massive stars, and SN~Ia.

\begin{verbatim}
s1.plot_mass(source='agb')
s1.plot_mass(source='massive')
s1.plot_mass(source='sn1a')

\end{verbatim}

Note that those lines appear all in one figure.
%The solar abundance distribution is taken from GN93 (NuGrid file).

%To get an overview over the available abundance distribution
%(in mass fraction divided by solar mass fraction) of the gas
%reservoir at a certain timestep, a function
%called plot\textunderscore abu\textunderscore distr() is available.
The function $plot\textunderscore mass\textunderscore range\textunderscore contributions()$
shows the yield contribution of stars as a function of their initial mass.
It allows to identify which mass range contributes to certain elements.
%Also plotted is to each yield (grid input) corresponding mass range
%as vertical lines. In these intervals the yields are the same (for one 
%simple stellar population).
To monitor the evolution of the mass of the gas reservoir ejected
masses over time, use plot\textunderscore totmasses().
The star formation rate as a function of time can be plotted with the function $plot\textunderscore sfr()$.
In the case you want to write out detailed chemical evolution tables
of isotopes, use the function $write\textunderscore evol\textunderscore table()$.
It writes out the mass of isotopes ejected by stars as a function of time
(each line represents a timestep).

\subsection{OMEGA}

OMEGA is a classical one-zone galaxy model with
an input star formation history
where stars form and inject new elements within
the same gas reservoir, using SYGMA to create a SSP at every timestep.
A complete description of OMEGA's input parameters and functions can
be found in the Sphinx documentation.  

OMEGA can mimic known local galaxies, such as Sculptor and Fornax, by
using their specific star formation history and current mass, which is taken
from the literature.  The code offers three different prescriptions for treating
gas inflows and outflows (see the OMEGA Userguide ipython notebook).

All the parameters associated with simple stellar populations are treated as
in the SYGMA module.


\subsection{Getting Started with OMEGA}

Steps what to do with omega.


inflows and outflow implementation;
star formation input ...

\subsection{Stellab}

Cool stuff about stellab here.

\section{Disclaimer}

		
\section{History} 
This document history complements the svn log.

\begin{tabular*}{\textwidth}{lll}
\hline
Authors & yymmdd & Comment \\
CR & 130926 & Update of links and details \\
\hline
CR & 160513 & creation \\
\end{tabular*}


\subsection{Contact}
If any bugs do appear or if there are any questions, please email critter@uvic.ca
% --------------- latex template below ---------------------------
\begin{verbatim}

\end{verbatim}


%\begin{figure}[htbp]
%   \centering
%%   \includegraphics[width=\textwidth]{layers.jpg} % 
%      \caption{}   \includegraphics[width=0.48\textwidth]{FIGURES/HRD90ms.png}  
%   \includegraphics[width=0.48\textwidth]{FIGURES/HRD150ms.png}  
%
%   \label{fig:one}
%\end{figure}
%
%\begin{equation}
%Y\_a = Y\_k + \sum\_{i \neq k} Y\_i
%\end{equation}
%
%{
%%\color{ForestGreen}
%\sffamily 
%  {\center  --------------- \hfill {\bf START: Some special text} \hfill ---------------}\\
%$Y\_c$ does not contain ZZZ but we may assign one $Y_n$ to XYZ which is the decay product of the unstable nitrogen isotope JJHJ. %
%
%{\center ---------------  \hfill {\bf END:Some special text} \hfill ---------------}\\
%}



